\subsection{Casting}

The \MgZnCa alloy must only be induction melted within a \textbf{boron nitride crucible}. Once the molten alloy is cast, the crucible may \textbf{NOT} be used again (I.E. melt the alloy within crucible, pore the alloy to cast, crucible no longer usable). 

\subsubsection{Ingot heating cycle}

This heating/cooling cycle helps insure a homogeneous alloy melt.
\begin{itemize}
\item The \MgZnCa  alloy ingots/pieces should be heated to $650^{\circ}C$ to a full melt
\item Once melted the alloy should be cooled to $385^{\circ}C$
\item The alloy should then be reheated to $650^{\circ}C$
\item The alloy should then be cooled below $500^{\circ}C$ and cast into the mould
\item Do NOT cast last liquid from the crucible as this is slag with bad properties. 
\end{itemize}

\subsubsection{Notes for induction melting of \MgZnCa alloy.}
\begin{itemize}
\item The \MgZnCa alloy melts around $400^{\circ}C$
\item Alloy surface should cleaned with a light sanding 320 grit sanding
\item ingot should be packed to maximise crucible ingot wall contact
\item small crucible holds 40cc, the large 60cc
\item If induction currents do not provide sufficent stirring, this can be accomplished manually via a small W wire. 
\item Return spent crucibles to UNSW for reservicing. 
\end{itemize}