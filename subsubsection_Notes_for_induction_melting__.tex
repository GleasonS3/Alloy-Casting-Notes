\subsubsection{Notes for induction melting of \MgZnCa alloy.}
\begin{itemize}
\item The \MgZnCa alloy melts around $400^{\circ}C$;
\item Melting should be under an inert atmosphere (UNSW uses Ar);
\item Surfaces of alloy ingot pieces should be lightly sanded with 320 grit paper to remove contamination before melting;
\item Ingot pieces should be packed to maximise crucible/ingot wall contact;
\item Crucible volumes are 40$cc$ (small), and 60$cc$ (large). About 114$g$ and 171$g$ respectively;
\item If induction currents do not provide sufficient stirring/mixing, this can be accomplished manually via a small W wire; 
\item Ingot pieces can be easily fractured via wrapping in a flexible polymer sheet (to capture projectiles) and lightly hitting with hammer;
\item Return spent crucibles to UNSW for reservicing. 
\end{itemize}

If there are any additional questions, please do not hesitate to contact me. 