\subsubsection{Notes for induction melting of \MgZnCa alloy}
\begin{itemize}
\item The \MgZnCa alloy's melt temperature ($T_{m}$) is about $335^{\circ}C$, and the liquidus temperature ($T_{l}$) is about $412^{\circ}C$;
\item Melting should be under an inert atmosphere (\textit{UNSW} uses Ar);
\item Surfaces of alloy ingot pieces should be lightly sanded with 320 grit paper to remove contamination before melting;
\item Ingot pieces should be arranged to maximise ingot/crucible wall contact;
\item Ingot pieces can be easily fractured via wrapping in a flexible polymer sheet (to capture projectiles) and 'lightly' hitting with a hammer;
\item Crucible working volumes are 40$cc$ (small), and 60$cc$ (large). About 114$g$ and 171$g$ of \MgZnCa respectively;
\item If induction currents do not provide sufficient stirring/mixing, it can be accomplished manually via careful stirring with a tungsten (W) wire; and
\item Please return spent crucibles to \textit{UNSW} for reservicing. 
\end{itemize}

If there are any additional questions, please do not hesitate to contact me at \href{"mailto:s.gleason@student.unsw.edu.au"}{S.Gleason@student.unsw.edu.au}. 